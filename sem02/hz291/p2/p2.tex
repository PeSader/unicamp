\documentclass[
    report,
    11pt,
    oneside,
    a4paper,
    english,
    brazil,
    sumario=tradicional
    ]{abntex2}


% Pacotes fundamentais
\usepackage{lmodern}
\usepackage[T1]{fontenc}
\usepackage[utf8]{inputenc}
\usepackage{indentfirst}
\usepackage{nomencl}
\usepackage{color}
\usepackage{microtype}
\usepackage{graphicx}
\graphicspath{{img/}}

% Pacotes adicionais
\usepackage{lipsum}
\usepackage[brazilian,hyperpageref]{backref}
\usepackage[alf]{abntex2cite}

% Configurações do pacote backref
\renewcommand{\backrefpagesname}{Citado na(s) página(s):~}
\renewcommand{\backref}{}
\renewcommand*{\backrefalt}[4]{
    \ifcase #1
        Nenhuma citação no texto.
    \or
        Citado na página #2.
    \else
        Citado #1 vezes nas páginas #2.
    \fi}%

\title{Os Efeitos Adversos da GDPR}
\tituloestrangeiro{\textbf{How the GDPR Failed} \\ \Large{And how we can still succeed}}

\autor{Pedro Sader Azevedo}
\instituicao{Universidade Estadual de Campinas}
\local{Brasil}

% Configurações de aparência do PDF final
% alterando o aspecto da cor azul
\definecolor{blue}{RGB}{41,5,195}

\makeatletter
\hypersetup{
   pagebackref=true,
   pdftitle={\@title},
   pdfauthor={\@author},
   pdfsubject={A complexidade do problema da desigualdade salarial entre homens e mulheres},
   pdfcreator={LaTeX with abnTeX2},
   pdfkeywords={ods}{gênero}{salário}{ocupação},
   colorlinks=true, % false: boxed links; true: colored links
   linkcolor=blue, % color of internal links
   citecolor=blue, % color of links to bibliography
   filecolor=magenta, % color of file links
   urlcolor=blue,
   bookmarksdepth=4
}
\makeatother

% compila o indice
\makeindex

% margens
\setlrmarginsandblock{3cm}{3cm}{*}
\setulmarginsandblock{3cm}{3cm}{*}
\checkandfixthelayout

% tamanho da indentação
\setlength{\parindent}{1.3cm}

% espaçamento entre parágrafos
\setlength{\parskip}{0.2cm}
% \setlength{\onelineskip}

\SingleSpacing


\begin{document}

\selectlanguage{brazil}

\frenchspacing

\maketitle

\textual

A Regulamentação Geral de Proteção de Dados (GDPR) foi a principal política pública europeia referente à questão da privacidade digital~\cite{gdpr}. Essa lei foi tomada como referência em diversas partes do mundo, inclusive servindo de base para a criação da Lei Geral de Proteção de Dados no Brasil~\cite{lgpd}. No entanto, as medidas estabelecidas pela GDPR provaram-se não apenas ineficazes em ampliar o respeito pelo Direito Humano à privacidade, como também introduziram novos empecilhos a conquista desse direito.

Uma das mais emblemáticas resoluções da GDPR foi a proibição da coleta de dados pessoais sem consentimento explícito~\cite{data-consent}. Apesar de ``bem-intencionada'', essa imposição teve como efeito adverso uma abundância desmedida de \textit{cookie popups}, isto é, mensagens que cobrem o conteúdo de páginas da web até que o usuário ceda a permissão para coleta e processamento de seus dados.

\begin{figure}[h]
    \includegraphics[width=\textwidth]{cookies.png}
    \centering
    \caption{A Internet depois da GDPR (autoral)}
\end{figure}

Consequência imediata disso foi uma queda dramática na qualidade da experiência de usuário, visto que interrupções constantes se tornaram regra (ou melhor, lei), em vez de excessão. Além disso, estudos apontam que mencionar ``personalização'' ou ``aprimoramento de serviços'' aumenta significativamente o poder persuasivo dos \textit{cookie banners}~\cite{cookie-strategies}, muitas vezes levando à decisão mal-informada de consentir.

Enfim, por mais que a GDPR não tenha sido perfeita, é inegável sua importância. No mínimo, ela desencadeou a tendência de regulamentar mais estritamente os mercado digitais~\cite{dma} e, por isso, merece crédito.

\pagebreak

\postextual

\bibliography{ref}

\end{document}
