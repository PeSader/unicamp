\documentclass[
    article,
    11pt,
    oneside,
    a4paper,
    english,
    brazil,
    sumario=tradicional
    ]{abntex2}


% Pacotes fundamentais
\usepackage{lmodern}
\usepackage[T1]{fontenc}
\usepackage[utf8]{inputenc}
\usepackage{indentfirst}
\usepackage{nomencl}
\usepackage{color}
\usepackage{graphicx}
\usepackage{microtype}

% Pacotes adicionais
\usepackage{lipsum}
\usepackage[brazilian,hyperpageref]{backref}
\usepackage[alf]{abntex2cite}

% Configurações do pacote backref
\renewcommand{\backrefpagesname}{Citado na(s) página(s):~}
\renewcommand{\backref}{}
\renewcommand*{\backrefalt}[4]{
	\ifcase #1 %
		Nenhuma citação no texto.%
	\or
		Citado na página #2.%
	\else
		Citado #1 vezes nas páginas #2.%
	\fi}%

% Informações para capa e folha de rosto
\titulo{A Enganosa Simplicidade do ODS 5}
\tituloestrangeiro{}

\autor{Pedro Sader Azevedo}
\local{Brasil}

% Configurações de aparência do PDF final
% alterando o aspecto da cor azul
\definecolor{blue}{RGB}{41,5,195}

% informações do PDF
\makeatletter
\hypersetup{
     	%pagebackref=true,
		pdftitle={\@title},
		pdfauthor={\@author},
    	pdfsubject={A complexidade do problema da desigualdade salarial entre homens e mulheres},
	    pdfcreator={LaTeX with abnTeX2},
		pdfkeywords={ods}{igualdade de gênero}{salário}{ocupação},
		colorlinks=true,       		% false: boxed links; true: colored links
    	linkcolor=blue,          	% color of internal links
    	citecolor=blue,        		% color of links to bibliography
    	filecolor=magenta,      		% color of file links
		urlcolor=blue,
		bookmarksdepth=4
}
\makeatother
% ---

% ---
% compila o indice
% ---
\makeindex
% ---

% ---
% Altera as margens padrões
% ---
\setlrmarginsandblock{3cm}{3cm}{*}
\setulmarginsandblock{3cm}{3cm}{*}
\checkandfixthelayout
% ---

% ---
% Espaçamentos entre linhas e parágrafos
% ---

% O tamanho do parágrafo é dado por:
\setlength{\parindent}{1.3cm}

% Controle do espaçamento entre um parágrafo e outro:
\setlength{\parskip}{0.2cm}  % tente também \onelineskip

% Espaçamento simples
\SingleSpacing


% ----
% Início do documento
% ----
\begin{document}

% Seleciona o idioma do documento (conforme pacotes do babel)
%\selectlanguage{english}
\selectlanguage{brazil}

% Retira espaço extra obsoleto entre as frases.
\frenchspacing

% ----------------------------------------------------------
% ELEMENTOS PRÉ-TEXTUAIS
% ----------------------------------------------------------
% página de titulo principal (obrigatório)
\maketitle

% ----------------------------------------------------------
% ELEMENTOS TEXTUAIS
% ----------------------------------------------------------
\textual

A Agenda 2030 é um documento internacional que contém dezessete Objetivos de
Desenvolvimento Sustentável (ODS), acordados pelos países membros das Nações
Unidas como imprescindíveis para a confluência dos aspectos ambientais, sociais
e econômicos da experiência humana~\cite{unitednations}. Sendo assim, é
imediato que a igualdade de gênero deve constar entre esses objetivos, o que de
fato ocorre no ODS 5~\cite{ods5}.

Curiosamente, esse foi o ODS mais frequentemente escolhido pelos alunos de
HZ291 para implementação em suas próprias esferas de influência (bairro,
universidade, empresa, etc). Eu mesmo escolhi esse objetivo, julgando-o mais
palpável que as demais alternativas. No contexto de contratação de
funcionários, por exemplo, bastaria ocultar a variável de gênero da comissão de
seleção para evitar qualquer viés potencialmente originado dessa
característica. Essa proposta também eliminaria as muitas dificuldades
introduzidas por sistemas de cotas, a começar pela decisão de quantos gêneros
considerar. Após uma reflexão mais cuidadosa, no entanto, percebi que a
influência dessa proposta simplista seria minúscula perante à imensidão do
problema da desigualdade de gênero.

No âmbito ocupacional, sabe-se que a participação feminina no mercado de
trabalho brasileiro tem crescido estavelmente apesar das flutuações
econômicas~\cite{Lavinas2001}. No entanto, mulheres são a demografia
predominante em cargos mal remunerados~\cite{Barros1997, Araujo2001} e
informais~\cite{Cristina2007}. Além disso, a criação da prole é ainda
desempenhada principalmente por mulheres, o que sobrecarrega severamente as
mães trabalhadoras~\cite{Cristina2007}. Isso significa que uma medida
verdadeiramente eficiente para nos aproximarmos do cumprimento do ODS 5 deve
ser holística, isto é, deve levar em conta os muitos aspectos que contribuem
para a desigualdade de gênero: desde a distribuição irregular da
responsabilidade pela maternidade-paternidade até as opções de carreira e
consequentes diferenças em poder aquisitivo.

\pagebreak

\postextual

\bibliography{ref}

\end{document}
